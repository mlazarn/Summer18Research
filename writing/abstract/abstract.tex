\documentclass{article}
\usepackage{amsmath}
\usepackage{amsfonts}
\usepackage{amssymb}
\usepackage{amsthm}
\usepackage{amsmath}
\usepackage{verbatim}
\usepackage{enumitem}
\usepackage[margin=1.25in]{geometry}
%\setlength{\parindent}{0cm}

\title{Boundary Effects in Stochastic Cyclic Competition Models on a Two-Dimensional Lattice}
\author{M. Lazarus Arnau, Shannon Serrao, Uwe C. T{\"a}uber\\ Department of Physics (MC 0435) and Center for Soft Matter and Biological Physics\\ Virginia Tech, Blacksburg, Virginia 24061}
%\date{July 26, 2018}

\begin{document}
    
    \maketitle
    We study noise-induced and -stabilized spatial patterns in two distinct stochastic 
    population model variants for cyclic competition of three species, namely the 
    Rock-Paper-Scissors (RPS) and the May-Leonard (ML) models. In two dimensions, 
    it is well established that the ML model can display (quasi-)stable spiral 
    structures, in contrast to simple species clustering in the RPS system. Our 
    ultimate goal is to impose control over such competing structures in systems 
    where both RPS and ML reactions are implemented. To this end, we have employed 
    Monte Carlo computer simulations to investigate how changing the microscopic 
    rules in a subsection of a two-dimensional lattice influences the macroscopic 
    behavior in the rest of the lattice. Specifically, we implement the ML reaction scheme 
    on a torus, except on a ring-shaped patch, which is set to follow the cyclic 
    Lotka-Volterra predation rules of the RPS model. There, we observe a marked disruption of 
    the usual spiral patterns in the form of plane waves emanating from the RPS region. 
    Using fast Fourier transforms we find that the spacial extent of these disruptions
    is set by the diffusion rate in the May-Leonard region. Furthermore, the overall 
    population density drops considerably in the vicinity of the interface between both regions.

    %\hskip 

    We will continue to characterize these effects before using our deepened understanding
    of these systems to develop possible methods for local external control. We plan to
    begin by exploring the possibility of using pulsed and periodic seeding to either
    stabilize or destroy the spiral patterns displayed by our simulations.

    \hskip

    Research was sponsored by the U.S. Army Research Office and was accomplished 
    under Grant Number W911NF-17-1-0156. 
\end{document}
