\documentclass[12pt]{article}
\usepackage{amsmath}
\usepackage{amsfonts}
\usepackage{amssymb}
\usepackage{amsthm}
\usepackage{amsmath}
\usepackage{verbatim}
\usepackage{enumitem}
\usepackage[margin=1.25in]{geometry}

\title{Plane-Waves at the Interface Between Two-Dimensional Rock-Paper-Scissors and May-Leonard Systems}
\author{M. Lazarus Arnau\\ PHYS 4316 Modern Experimental Physics}

\begin{document}
    
\maketitle

The Rock-Paper-Scissors (RPS) and May-Leonard (ML) models are two models are used 
to study competition and co-existance in systems which obey a cyclic (or 
"rock-paper-scissors") predation scheme. When implemented on a two-dimensional 
lattice, the RPS model exhibits clustering of species whereas the ML system, under
certain parameter regimes, can produce long-lived spiral-wave formations. This 
difference in behavior is a result of the fact that the RPS model treats predation 
and reproduction as a single process, where in the ML model predation and 
reproduction are two seperate processes. The formation of spirals in ML systems 
is an example of a noise-induced nonlinear effect. 

In research performed with Prof. Uwe T{\"a}uber and Shannon Serrao we used 
Monte-Carlo techniques to simulate a two-dimensional lattice with periodic 
boundary conditions which obeys the microscopic rules of the ML model everywhere
except for a narrow vertical band which obeys the rules of the RPS model. We 
discovered that, near the interface, the spiral-waves gave way to plane-waves
which travel away from the boundary in to the ML region. We also observed a marked
drop in population density near the interface.
%We study noise-induced and -stabilized spatial patterns in two distinct stochastic 
%population model variants for cyclic competition of three species, namely the 
%Rock-Paper-Scissors (RPS) and the May-Leonard (ML) models. In two dimensions, it 
%is well established that the ML model can display (quasi-)stable spiral structures, 
%in contrast to simple species clustering in the RPS system. Our ultimate goal is 
%to impose control over such competing structures in systems where both RPS and ML 
%reactions are implemented. To this end, we have employed Monte Carlo computer 
%simulations to investigate how changing the microscopic rules in a subsection of 
%a two-dimensional lattice influences the macroscopic behavior in the rest of the 
%lattice.  Specifically, we implement the ML reaction scheme on a torus, except on 
%a ring-shaped patch, which is set to follow the cyclic Lotka-Volterra predation 
%rules of the RPS model. There, we observe a marked disruption of the usual spiral 
%patterns in the form of plane-waves emanating from the RPS region. 

This semester we will continue our investigation of this phenomenon. We will 
measure to what extent the presence and strength of the plane-waves are dependent 
on the mobility rate in the RPS and ML regions. We hope to use this information 
to investigate how local periodic seeding and/or mixing can be used to disrupt 
the formation of otherwise stable patterns in ML systems. 


\begin{thebibliography}{99}

    \bibitem{pres} M. Lazarus Arnau, Shannon Serrao, Uwe C. T{\"a}uber, \emph{}Boundary Effects in Stochastic Cyclic Competition Models on a Two-Dimensional Lattice, (Wiley-VCH, 2008, Windheim) (ISBN: 978-0-471-60386-3)

\end{thebibliography}
\end{document}
